\documentclass[11pt]{report}
\usepackage[utf8]{inputenc}
\usepackage{geometry}
\geometry{
a4paper,
total = {160mm,245mm},
left =30mm,
top = 30mm,
}
\pagenumbering{Alph}

\begin{document}
\begin{titlepage}
   \begin{center}
       \vspace*{5cm}

       \textbf{Development Tools Laboratory}

       \vspace{0.5cm}
        Lab Course
            
       \vspace{1.5cm}

       \textbf{Ashwini Mekhe}

       \vfill
            
       A course presented for B.Tech Computer Science Students\\
            
       \vspace{0.8cm}
     
       
            
       Computer Engineering\\
       COEP Technological University\\
       Pune, Maharashtra\\
       Date
            
   \end{center}
\end{titlepage}
\clearpage
\chapter{Introduction to LATEX}
\section{Introduction}
LaTeX is a software system for document preparation.
\subsection{History}
LaTeX was created in the early 1980s by Leslie Lamport, when he was working at SRI. He needed to write TeX macros for his own use, and thought that with a little extra effort he could make a general package usable by others. Peter Gordon, an editor at Addison-Wesley, convinced him to write a LaTeX user's manual for publication it came out in 1986 and sold hundreds of thousands of copies.
\subsubsection{Typesetting System}
LaTeX attempts to follow the design philosophy of separating presentation from content, so that authors can focus on the content of what they are writing without attending simultaneously to its visual appearance. In preparing a LaTeX document, the author specifies the logical structure using simple, familiar concepts such as chapter, section, table, figure, etc., and lets the LaTeX system handle the formatting and layout of these structures. As a result, it encourages the separation of the layout from the content — while still allowing manual typesetting adjustments whenever needed. This concept is similar to the mechanism by which many word processors allow styles to be defined globally for an entire document, or the use of Cascading Style Sheets in styling HTML documents.
\paragraph{As a macro package, LaTeX provides a set of macros for TeX to interpret. There are many other macro packages for TeX, including Plain TeX, GNU Texinfo, AMSTeX, and ConTeXt. When TeX "compiles" a document, it follows the following processing sequence: Macros → TeX → Driver → Output. Different implementations of each of these steps are typically available in TeX distributions. Traditional TeX will output a DVI file, which is usually converted to a PostScript file.}
\clearpage
\setcounter{page}{3}
\pagenumbering{roman}
\chapter{Advantages of LATEX} 

\end{document}